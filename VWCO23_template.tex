\documentclass{article}
\usepackage{amsmath,amssymb,amsthm,amsfonts,amsopn,cite,mathrsfs}
\begin{document}
\def\title#1{#1}
\def\author#1{#1}
\def\affil#1{#1}
\def\country#1{#1}
\def\email#1{#1}
\def\jointwith#1{#1}

%%%!DO NOT EDIT ANYTHING ABOVE THIS LINE!------------------------------------

\begin{center}
\title{{\large Title of the presentation}}\\[2ex]
\author{Name}\\
\affil{Affiliation}\\
\email{Email address}\\[2ex]
\jointwith{joint work with Coauthor 1, Coauthor 2, ...}
\end{center}

\begin{abstract}

The \textit{ab initio} quantum chemical computation is a highly time consuming process. In recent times, machine learning techniques have been proven to be flexible and efficient in approximating functions and are widely used in many fields, including quantum chemistry. Machine learning predictors are several orders of magnitude faster than the \textit{ab initio} class of methods....

\begin{itemize}
	\item Motivation
	\item Problem statement
	\item Approach
	\item Results
	\item Conclusions
\end{itemize}
https://users.ece.cmu.edu/~koopman/essays/abstract.html
\begin{thebibliography}{7}

\bibitem{ref} Name, \textit{Title}, Journal.~\textbf{5} (2022), 101--124.


\end{thebibliography}

%%%%!DO NOT EDIT ANYTHING BELOW THIS LINE!------------------------------------


\end{abstract}

\end{document}