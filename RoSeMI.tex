\documentclass[12pt]{article}

% Language setting
% Replace `english' with e.g. `spanish' to change the document language
\usepackage[english]{babel}

% Set page size and margins
% Replace `letterpaper' with `a4paper' for UK/EU standard size
\usepackage[a4paper,top=2cm,bottom=2cm,left=3cm,right=3cm,marginparwidth=1.75cm]{geometry}

% Useful packages
\usepackage{amsmath}
\usepackage{amssymb}
\usepackage{graphicx}
\usepackage{mathtools}
\usepackage{enumitem}
\usepackage{algorithm}
\usepackage{algorithmicx}
\usepackage{algpseudocode}
\usepackage[colorlinks=true, allcolors=blue]{hyperref}
\usepackage{caption}
\captionsetup[table]{skip=10pt} % table skip

%macros
\newcommand{\defeq}[2]{\stackrel{\mathclap{\normalfont\mbox{#1}}}{#2}}
\def\D{\displaystyle}
\def\att{                    % mark at the margin
        \marginpar[ \hspace*{\fill} \raisebox{-0.2em}{\rule{2mm}{1.2em}} ]
        {\raisebox{-0.2em}{\rule{2mm}{1.2em}} }
        }
\def\at#1{[*** \att #1 ***]}  % text needing attention
\def\spc{\hspace*{0.5cm}}



\title{RoSeMI: Robust Shepard model for interpolation}
%\author{Beryl Ramadhian Aribowo}

\begin{document}
\maketitle


\section{The model}
The generalized Shepard model has the form of 
\begin{equation}
    \label{eq:shepard}
    V_K(w):=R_K(w)/S_K(w) ~~~ \text{ for } w \in \mathbb{R}^m \setminus \{w_k\mid k\in K \},
\end{equation}
where
\begin{equation}
    R_K(w):=\sum_{k\in K} \frac{V_k(w)}{D_k(w)},~~~
    S_K(w):=\sum_{k\in K} \frac{1}{D_k(w)},
\end{equation}
and
\begin{equation}
    \label{eq:vk}
    V_k(w) = E_k + \sum_l \theta_{kl} \phi_{kl}(w).
\end{equation}
By (\ref{eq:vk}), (\ref{eq:shepard}) can be expanded into
\begin{equation}
    \label{eq:vk_expand}
    V_K(w) := \sum_{k\in K} \frac{E_k + \sum_l \theta_{kl} \phi_{kl}(w)}{D_k(w)} / S_K(w). 
\end{equation}
The quality of the prediction accuracy is measured by the \textbf{mean absolute deviation}
\begin{equation}
    \text{MAD}_K(w) := \frac{1}{|K|}\sum_{j\in K}|\Delta_{jK}(w)|
\end{equation}
and the \textbf{root mean square deviation}
\begin{equation}
    \text{RMSD}_K(w) := \sqrt{\frac{1}{|K|}\sum_{j\in K}\Delta_{jK}(w)^2},
\end{equation}
where by (\refeq{eq:vk}) and (\refeq{eq:vk_expand})
\begin{equation}
    \label{eq:delta}
    \begin{split}
        \Delta_{jK}(w)&:=\D\frac{V_K(w)-V_j(w)}{D_j(w)S_K(w)-1} \\
        &= \frac{\D \left(\sum_{k\in K} \frac{E_k + \sum_l \theta_{kl} \phi_{kl}(w)}{D_k(w)} / S_K(w)\right) - \left(E_j + \sum_l \theta_{jl} \phi_{jl}(w)\right)}{D_j(w)S_K(w)-1}, \\
    \end{split}
\end{equation}
and $l=1,2,...L$.

\section{Feature extraction}
The feature matrix $W \in \mathbb{R}^{n_\text{data} \times n_f}$ needs to be pre-computed and stored in the non-volatile storage in order to avoid re-computation. Below is roughly the flow of feature extraction used for fitting the QM9 dataset.
\begin{itemize}
    \item Compute the molecular features using ACSF (\at{cite Behler}). This would give a vector with length $= 51$ (i.e., $n^\text{atom}_f = 51$) for each atom. Apply this to all molecules in the dataset hence there should be a total of $n_\text{data}$ ACSF extractions.
    \item For each molecule containing $n_\text{atom}$ number of atoms, symmetrize the features by
        \begin{equation}
			\label{eq:acsf}
            \begin{split}
                F_{S_j} &:= \sum_{i=1}^{n_{\text{atom}}} F_{j,i},\\
                F_{Q_j} &:= \sum_{i=1}^{n_{\text{atom}}} F_{j,i}^2, \\
                \text{for }j &= 1,2,..., n^\text{atom}_f. 
            \end{split}
        \end{equation}
        then concatenating the vectors such that
        \begin{equation}
            F := (F_{S_1}, F_{S_2}, ..., F_{S_{n^\text{atom}_f}}, F_{Q_1},F_{Q_2} ..., F_{Q_{n^\text{atom}_f}}) \in \mathbb{R}^{n_f}.
        \end{equation}
    	then we stack the feature vector $F$ of each molecule as a row in the $W \in \mathbb{R}^{n_\text{data} \times n_f}$ matrix.
    \item Do feature selection to take the subset of the columns of $W$, this can be done by the following:
	    \begin{equation}
			\label{eq:pca_1}
			\begin{split}
				\hat{s} &= \sum_i W_{i,:}, \\
				S &= \sum_i W_{i,:}W_{i,:}^\top, \\
				\text{ for } i &= 1,2,...,n_{\text{data}}. \\
			\end{split}
	    \end{equation}
		then we compute the mean feature vector and covariance matrix
		\begin{equation}
			\begin{split}
				\bar{u} &= \hat{s}/n_\text{data}, \\
				C &= S/n_\text{data} - \bar{u}\bar{u}^\top. \\			
			\end{split}
		\end{equation}
		We select the eigenvectors such that they correspond to the $n_\text{select}$ largest eigenvalues, they are obtained by first doing the spectral decomposition of the covariance matrix
		\begin{equation}
			C = Q\Lambda Q^\top,
		\end{equation}
		where $\Lambda$ is the diagonal matrix containing the $k$th eigenvalue $\Lambda_{kk}$ and $Q$ is the matrix containing the $k$th eigenvector $Q_{:k}$, then we permute the columns of $Q$ and the entries of $\Lambda$ such that
		\begin{equation}
			\Lambda_{1,1} \geq \Lambda_{2,2} \geq  ... \geq \Lambda_{n_f,n_f},
		\end{equation}
		and select
		\begin{equation}
			\hat{Q} = Q_{:, 1:n_\text{select}}.
		\end{equation}
		Finally the transformed feature can be obtained by
		\begin{equation}
			\label{eq:pca_8}
			W_{i,:} := \hat{Q}^\top(W_{i,:} - \bar{u}), \text{ for }i = 1,2,...,n_\text{data}.
		\end{equation}
	\item Scale $W$ such that 
	\begin{equation}
		\label{eq:scale_1}
		W_{i,j} \in [0, 1]
	\end{equation}
	by
	\begin{equation}
		\label{eq:scale_2}
		\begin{split}
			W_{:,j} \leftarrow (W_{:,j} - \underbar{$W_{:,j}$})/ (\overline{W_{:,j}} - \underbar{$W_{:,j}$}), \text{ for } j = 1,2,..., n_f.
		\end{split}
	\end{equation}
	where \underbar{$W_{:,j}$} is the smallest value of the $j$th feature and $\overline{W_{:,j}}$ is the largest.
	\item Store $W$, the storage should be bounded by $O(N)$ for most cases unless much more features are needed (although currently from experiment this causes overfitting).
\end{itemize}

\section{Data selection}
The set of indices which refers to data points with known energies (or training data points) $K$ is chosen by the farthest-distance-algorithm variations, where the imposed challenge is $M = |K| = 100$. The flow of the data selection is as the following:
\begin{itemize}
	\item Compute and store distance matrix $D$ given $W$
	\begin{equation}
		D := f_{\text{dist}}(B, W) \in \mathbb{R}^{n_\text{data} \times n_\text{data}},
	\end{equation}
	where $D$ is symmetric with zero diagonal \at{might be necessary to be stored in sparse (reduces the matrix size, instead of $n \times n$ it becomes $n(n-1)/2$) for full computation of 113k molecules}, and $B$ is a matrix which transforms the feature space (currently $B := I$). The distance between two data points of index $k$ and $m$ can also be defined by
	\begin{equation}
		D_k(W_{m,:}) := \|W_{m,:} - W_{k,:}\|^2_B := \|B(W_{m,:} - W_{k,:})\|^2_2
	\end{equation}
	\item Compute and store the set of indices $K$ by farthest-minimum-distance algorithm \at{cite Eldar} given $D$, $B$ and $M$.
	\begin{equation}
		K := f_{\text{fmd}}(D, B, M) \in \mathbb{Z}^{+M},
	\end{equation}
	for example, we can obtain one data point which belongs to the training set by indexing
	\begin{equation}
		w_k := W_{k,:}, \text{ for } k \in K.
	\end{equation}
\end{itemize}

\section{Intermediate values}
\label{sec:intermediate}
Several intermediate values need to be computed and stored in the random access memory (RAM), since these intermediate values depend on several parameters which changes for each experiment and are fast to compute hence it is not necessary to store them in the non-volatile memory. Given the stored $W, D$ and $K$:
\begin{itemize}
	\item Compute the set of indices of the data points with unknown energy (or the test set):
	\begin{equation}
		T := \{1,2,...,n_\text{data}\} \setminus K.
	\end{equation}
	\item Compute intermediate values:
	\begin{equation}
		\label{eq:intermediate}
		\begin{split}
			S_K &\in \mathbb{R}^{T} := S_K(W_{m,:}) \text{ for } m \in T,\\
			\gamma &\in \mathbb{R}^{T \times M} := \gamma_k(W_{m,:}) \text{ for } k \in K, m \in T, \\
			\alpha &:= \gamma - 1.
		\end{split}
	\end{equation}
	\item Compute the univariate splines $\beta$ and its derivative $\beta'$ given $W_{i,j}$ for $i = 1,...,n_\text{data}$, $j = 1,...,n_f$ (vectorized over matrix $W$ for the computation of the spline and its derivative):
	\begin{equation}
		\begin{split}
			\Phi := \beta(W) \in \mathbb{R}^{n_\text{data} \times L}, \\
			\Phi' := \beta'(W) \in \mathbb{R}^{n_\text{data} \times L}.
		\end{split}
	\end{equation}
	where $L = n_fn_b$ and $n_b$ is the number of splines for each feature.
	\item Compute
	\begin{equation} % ϕ[l,m] - ϕ[l, k] - dϕ[l, k]*(W[t,m]-W[t,k])
		\phi_{m, kl} := \Phi_{m,l} - \Phi_{k,l} - \Phi'_{k,l}(W_{m,t}-W_{k,t}) \quad \text{for } m \in T, \text{ }k \in K, \text{ }l = 1,...,L,
	\end{equation}
	where
	\begin{equation}
		\begin{split}
			\hat{t} &:= l \text{ mod } n_f, \\
			t &\leftarrow \hat{t} = 0 \text{ ? } n_f : \hat{t},
		\end{split}
	\end{equation}
	this results in matrix $\phi$ where the block column is indexed by $k,l$ indices.
\end{itemize}

\section{Fitting the model}
Now that the required pieces are in place, we can fit the RoSeMI. First we need to express (\refeq{eq:delta}) as a system of linear equations, let us simplify it by writing
\begin{equation*}
	\label{eq:simp}
    \begin{split}
        D_k &:= D_k(w), \\
        S_K &:= S_K(w), \\
        \phi_{kl} &:= \phi_{kl}(w),\\
        \psi_{kl} &:= \theta_{kl}\phi_{kl}, \\
        \gamma_k &:= D_kS_K, \\
        \alpha_j &:= \gamma_j-1.
    \end{split}
\end{equation*}
using this, (\refeq{eq:delta}) can be re-arranged into
\begin{equation}
    \label{eq:delta_split}
    \begin{split}
        \Delta_{jK}(w)&= \frac{\D \left(\sum_{k\in K} \frac{E_k + \sum_l \psi_{kl}}{D_k} \right)/ S_K - \big(E_j + \sum_l \psi_{jl}\big)}{\alpha_j} \\
        &= \sum_k \frac{E_k}{\gamma_k\alpha_j} + \sum_{kl} \frac{\psi_{kl}}{\gamma_k\alpha_j} - \sum_{l} \frac{\psi_{jl}}{\alpha_j} - \frac{E_j}{\alpha_j}\\
        &= \sum_k \frac{E_k}{\gamma_k\alpha_j} + \sum_{kl} \frac{\psi_{kl}}{\gamma_k\alpha_j} - \sum_{l} \frac{\delta_{jk}\psi_{kl}}{\alpha_j} - \frac{E_j}{\alpha_j}\\
        &= \sum_{kl} \frac{\psi_{kl} (1-\gamma_k\delta_{jk})}{\gamma_k\alpha_j} - \left(E_j - \sum_k\frac{E_k}{\gamma_k}\right)/\alpha_j,
    \end{split}
\end{equation}
where $\delta_{jk}$ is the Kronecker delta. Hence if we pick $w = w_m$, then
\begin{equation}
	\begin{split}
		\Delta_{jK}(w_m) &= \sum_{k,l} A_{jm, kl} \theta_{kl} - b_{jm}, \\
	\end{split}	
\end{equation}
where
\begin{equation}
	\begin{split}
		A_{jm, kl} &= \frac{\phi_{kl}(w_m)(1 - \gamma_k(w_m) \delta_{jk})}{\gamma_k(w_m) \alpha_j(w_m)},\\
		b_{jm} &= \left(E_j - \sum_k\frac{E_k}{\gamma_k(w_m)}\right)/\alpha_j(w_m).
	\end{split}
\end{equation}


A good prediction model is reflected by small MAD or RMSD, hence a minimization routine is required. In order to solve the minimzation problem, a linear sysem or least squares formulation is needed. For example, if the RMSD is chosen as the objective function then the problem formulation would be
\begin{equation}
    \label{eq:min}
    \begin{split}
        \min |K| \sum_m \text{RMSD}_K(w_m)^2 &= \min \sum_m \sum_j \Delta_{jK}(w_m)^2 \\
        &= \min \sum_{jm} \left( A_{jm}\theta - b_{jm}\right)^2 \\
        f_{\text{obj}} &:= \min \|A \theta - b\|_2^2,
    \end{split}
\end{equation}
the data matrix $A$ with $MN$ rows and $ML$ columns where $N = |T|$, the coefficient vector $\theta$, and the target vector $b$.

For storage efficiency, it is desirable to not explicitly form the data matrix $A$, since it grows quickly with more data points included in the training set $K$ and testing set $T$. 
We can alleviate the storage problem by writing a routine for computing $Au$ where $u$ is a variable vector, hence instead of having a matrix with $\mathcal{O}(M^2N)$ storage we would have a vector with only $\mathcal{O}(N)$ storage. 
This can be done by augmenting the first sum of the last line of (\refeq{eq:delta_split}) into
\begin{equation}
	\label{eq:Au}
	\begin{split}
		Au = \sum_{kl} \frac{\phi_{kl}(w_m) u_{kl} (1-\gamma_k(w_m)\delta_{jk})}{\gamma_k(w_m)\alpha_j(w_m)},
	\end{split}
\end{equation}
or more explicitly, with the pre-computed intermediate values in (\refeq{eq:intermediate}):
\begin{equation}
	Au = \sum_{kl} \frac{\phi_{m, kl} u_{kl} (1-\gamma_{m, k}\delta_{jk})}{\gamma_{m,k}\alpha_{m,j}},
\end{equation}
Here, CGLS \at{cite Hestenes} method is used to minimize the objective function in (\refeq{eq:min}). The memory-less form of CGLS requires not only $Au$ but also $A^\top v$ routine, which is
\begin{equation}
	\begin{split}
		A^\top v = \sum_{jm} \frac{\phi_{m, kl} v_{jm} (1-\gamma_{m, k}\delta_{jk})}{\gamma_{m,k}\alpha_{m,j}}.
	\end{split}
\end{equation}
Given the $Au$ and $A^\top v$ routine, and $b$, the CGLS is called as follows \at{cite Krylov.jl and LinearOperators.jl}:
\begin{equation*}
	\begin{split}
		\text{op} &\leftarrow \text{LinearOperators}(y, Au, A^\top v), \\
		\theta &\leftarrow \text{CGLS}(\text{op}, b, \text{itmax}),
	\end{split}
\end{equation*}
where $y$ is an arbitrary vector and itmax is the maximum number of iteration.

\section{Numerical experiment}
Currently, all experiments are run on several hand-picked compounds, where each compound has $n_\text{data} \geq 250$. 
Each fitting is limited to each compound type. The CGLS used is set to 500 maximum iteration for each fitting, and only returns the parameter vector $\theta$ at the end.
The true error is described by the mean absolute error (MAE)
\begin{equation}
	\text{MAE} := \frac{1}{N}\sum_m|E_{m} - V_K(w_m)|.
\end{equation}
The result of the fitting for each compound is shown in table \ref{tab:exp}. The pre-computation time column refers only to the computation time of the intermediate values in section \ref{sec:intermediate}. 

Extensive experiments for H7C8N1 compound were also done to observe the effect of the features and bases, which is shown in table \ref{tab:exp_2}. Given a feature vector $F \in \mathbb{R}^{n_f} $ after PCA process (\refeq{eq:pca_1})-(\refeq{eq:pca_8}) and before scaling (hence actually $F:=W_{i,:}$), the binomial features refers to
\begin{equation}
	B_{p} := F_{j}F_{k} ~ \text{ for } j<k, ~ p = 1,...,n_f(n_f-1)/2,
\end{equation}
then the feature vectors are concatenated such that
\begin{equation}
	\hat{F} \leftarrow [F, B] \in \mathbb{R}^{n_f+n_f(n_f-1)/2},
\end{equation}
the "sum" variant refers to $F := F_S$ choice, and "sum(squared)" refers to $F := F_Q$ choice as in (\refeq{eq:acsf}). Then the feature vectors are scaled afterwards following (\refeq{eq:scale_1}) and (\refeq{eq:scale_2}).


\begin{table}[h]
	\caption{Fitting for various compounds.}
	\label{tab:exp}
	\begin{tabular}{|c|c|c|c|c|c|c|c|c|}
	\hline
	\textbf{Compound} & \textbf{$M$} & \textbf{$N$} & \textbf{$n_f$} & \textbf{$n_b$} & \textbf{MAE (kcal/mol)} & \textbf{$f_\text{obj}$} & \textbf{\begin{tabular}[c]{@{}c@{}}Pre-comp.\\  time (s)\end{tabular}} & \textbf{\begin{tabular}[c]{@{}c@{}}CGLS \\ time (s)\end{tabular}} \\ \hline
	H6C5N4            & 100          & 150          & 40             & 15             & 8.298e+00               & 4.334e-08               & 0.395                                                                  & 1636.52                                                           \\ \hline
	H6C5O3            & 100          & 150          & 40             & 15             & 7.199e+00               & 4.145e-08               & 0.399                                                                  & 1615.242                                                          \\ \hline
	H6C5O4            & 100          & 150          & 40             & 15             & 7.008e+00               & 3.167e-08               & 0.395                                                                  & 1623.974                                                          \\ \hline
	H6C6O2            & 100          & 150          & 40             & 15             & 7.225e+00               & 5.235e-08               & 0.606                                                                  & 1641.69                                                           \\ \hline
	H6C6O3            & 100          & 150          & 40             & 15             & 8.963e+00               & 5.501e-08               & 0.507                                                                  & 1614.349                                                          \\ \hline
	H6C7N2            & 100          & 150          & 40             & 15             & 6.605e+00               & 5.583e-08               & 0.391                                                                  & 1630.287                                                          \\ \hline
	H6C7O2            & 100          & 150          & 40             & 15             & 7.67e+00                & 5.28e-08                & 0.581                                                                  & 1616.849                                                          \\ \hline
	H7C6N3            & 100          & 150          & 40             & 15             & 5.229e+00               & 3.513e-08               & 0.395                                                                  & 1628.509                                                          \\ \hline
	H7C8N1            & 100          & 150          & 40             & 15             & 7.442e+00               & 6.475e-08               & 0.4                                                                    & 1620.605                                                          \\ \hline
	H10C9             & 100          & 150          & 40             & 15             & 8.849e+00               & 2.347e-08               & 0.416                                                                  & 1632.832                                                          \\ \hline
	H12C8             & 100          & 150          & 40             & 15             & 6.719e+00               & 6.701e-09               & 0.417                                                                  & 1631.973                                                          \\ \hline
	H12C9             & 100          & 150          & 40             & 15             & 8.793e+00               & 8.396e-09               & 0.592                                                                  & 1629.829                                                          \\ \hline
	H16C9             & 100          & 150          & 40             & 15             & 1.05e+01                & 1.082e-10               & 0.428                                                                  & 1655.233                                                          \\ \hline
	\end{tabular}
\end{table}

\begin{table}[h]
	\caption{Fitting for H7C8N1 compound only, with varied features and bases.}
	\label{tab:exp_2}
	\begin{tabular}{|c|c|c|c|c|c|c|c|c|}
	\hline
	\textbf{$M$} & \textbf{$N$} & \textbf{$n_f$} & \textbf{$n_b$} & \textbf{\begin{tabular}[c]{@{}c@{}}MAE \\ (kcal/mol)\end{tabular}} & \textbf{$f_\text{obj}$} & \textbf{\begin{tabular}[c]{@{}c@{}}Pre-comp.\\ time (s)\end{tabular}} & \textbf{\begin{tabular}[c]{@{}c@{}}CGLS\\ time (s)\end{tabular}} & \textbf{\begin{tabular}[c]{@{}c@{}}feature \\ type\end{tabular}}          \\ \hline
	100          & 150          & 24             & 8              & 6.079e+00                                                          & 1.208e-06               & 0.123                                                                 & 591.844                                                          & default                                                                   \\ \hline
	100          & 150          & 28             & 8              & 4.529e+00                                                          & 5.199e-07               & 0.142                                                                 & 747.534                                                          & default                                                                   \\ \hline
	100          & 150          & 21             & 8              & 6.862e+00                                                          & 2.107e-06               & 0.105                                                                 & 517.05                                                           & default                                                                   \\ \hline
	100          & 150          & 21             & 8              & 1.287e+01                                                          & 5.631e-05               & 0.112                                                                 & 513.35                                                           & \begin{tabular}[c]{@{}c@{}}binomial(\\ PCA(sum))\end{tabular}          \\ \hline
	100          & 150          & 21             & 8              & 9.516e+00                                                          & 1.099e-04               & 0.099                                                                 & 523.964                                                          & \begin{tabular}[c]{@{}c@{}}binomial(\\ PCA(sum(squared)))\end{tabular} \\ \hline
	100          & 150          & 102            & 8              & 8.177e+00                                                          & 6.956e-09               & 0.643                                                                 & 2333.708                                                         & default                                                                   \\ \hline
	100          & 150          & 102            & 15             & 1.963e+01                                                          & 1.32e-09                & 1.162                                                                 & 4122.527                                                         & default                                                                   \\ \hline
	100          & 150          & 40             & 15             & 7.442e+00                                                          & 6.475e-08               & 0.401                                                                 & 1620.606                                                         & default                                                                   \\ \hline
	\end{tabular}
	\end{table}

\iffalse
========= some comments========== \\
$b$ must of course be independent of $\theta$.

In (8) there is no argument $w$; instead you need to minimize the expression
\begin{equation}
    |K| \sum_m RMSD_K(w_m)^2,
\end{equation}
where $m$ labels the molecules. To find the linear system you need to write the right hand side of (7) as
\begin{equation}
    \sum_{kl} A_j(w)_{kl} \theta_{kl} -b_j(w)
\end{equation}
and find the expressions for $A_j(w)_{kl}$ and $b_j(w)$ by comparing the coefficients of $\theta_{kl}$ and the $\theta$-independent terms. For $A_j(w)_{kl}$ you get a sum of two terms, one involving a Kronecker delta. For $b_j(w)$ you get a sum over $k$ and an additional term.
The least squares problem in (8) then has a matrix $A$ with rows indexed by $j,m$ and columns indexed by $k,l$, with
\begin{equation}
    A_{jm,kl}:=A_j(w_m)_{kl},
\end{equation}
and something similar for the entries of $b$.

================== \\
You just need to extract from (7) the terms containing $\theta_{kl}$, which are

\begin{equation}
    \frac{\phi_{kl}}{D_kS_K(D_jS_K-1)}
\end{equation}


and
\begin{equation}
    \frac{\delta_{jk}\phi_{kl}D_kS_K(D_jS_K-1)}{D_jS_K-1}.
\end{equation}


Add the two and simplify, taking out common factors.

Similarly, you extract the terms without theta's by setting all $\theta$'s to zero

================== \\

Please write a program for calculating the vector v with components $v_{jm}:=\Delta_{jK}(w_m)$ using (3), (2), (1), and (7a) of your writeup, adding in (3) only the nonzero terms. Compare times with those for computing $A\theta-b$. The numerical values should be the same, up to roundoff.

\fi

\end{document}
