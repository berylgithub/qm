\documentclass{article}
\usepackage{amsmath,amssymb,amsthm,amsfonts,amsopn,cite,mathrsfs}
\begin{document}
\def\title#1{#1}
\def\author#1{#1}
\def\affil#1{#1}
\def\country#1{#1}
\def\email#1{#1}
\def\jointwith#1{#1}

%%%!DO NOT EDIT ANYTHING ABOVE THIS LINE!------------------------------------

\begin{center}
\title{{\large Optimization in data science: 100 energies to learn quantum chemistry}}\\[2ex]
\author{Beryl Ramadhian Aribowo}\\
\affil{Affiliation}\\
\email{Email address}\\[2ex]
\jointwith{joint work with Arnold Neumaier}
\end{center}

\begin{abstract}

The Schr\"{o}dinger equation is a powerful tool to explain the atomic world, it can be solved by \textit{ab initio} quantum chemistry methods accurately; however, they require massive computational resources. 
In recent times, various machine learning models coupled with optimization methods have been proven to be flexible and efficient in approximating functions and are widely used in many fields, including quantum chemistry. 
Given an optimized machine learning model, it can predict quantum properties of molecules in several orders of magnitude faster than the \textit{ab initio} methods. 
Here we present machine learning models which are simple to optimize and only have small computational cost.
Machine learning models such as a combined supervised--unsupervised kernel and Gaussian kernel on molecular and atomic level are employed. 
Despite the simplicity, finding the best model configuration remains a challenge, hence we also employ an efficient hyperparameter optimization scheme using a derivative free optimizer. 
The QM9 dataset is chosen as a benchmark for the machine learning techniques. With only 100 samples for training, 1 kcal/mol chemical accuracy is almost achieved across the whole 130k test samples.


\iffalse
\begin{itemize}
	\item Motivation
	\item Problem statement
	\item Approach
	\item Results
	\item Conclusions
\end{itemize}
https://users.ece.cmu.edu/~koopman/essays/abstract.html
\fi

\begin{thebibliography}{7}

\bibitem{ref} Name, \textit{Title}, Journal.~\textbf{5} (2022), 101--124.


\end{thebibliography}

%%%%!DO NOT EDIT ANYTHING BELOW THIS LINE!------------------------------------


\end{abstract}

\end{document}